\section{Lineare Algebra / Matrixexponential}
%TODO by Selina
-Transformationen\\
-Jordan Normalform\\
-Matrixexponential 

\subsection{Berechnung Jordan Normalform mit TI89}
Matrix:
$\begin{bmatrix} %phantom is for spacing
	2 & \phantom{-}0 & \phantom{-}5\\
	5 & -3 & \phantom{-}6\\
	0 & \phantom{-}0 & -3\\
\end{bmatrix}$

Step 1: Save Matrix\\{}
[2,0,5;5,-3,6;0,0,-3] $\rightarrow$ a\\
\hspace*{2cm}$\begin{bmatrix} %phantom is for spacing
	2 & 0 & 5\\
	5 & -3 & 6\\
	0 & 0 & -3\\
\end{bmatrix}$

Step 2: Eigenwerte\\
eigVl(a)\\
\hspace*{2cm} \{-3,2,-3\}

Step 3: Basisvektor $b_1 \rightarrow (A- \lambda_1 I)^n = 0$ für $n=\{1\}$\\
\textcircled{1} rref(A-2*identity(3))\\
\hspace*{2cm}$\begin{bmatrix}
	1 & -1 & 0\\
	0 & 0 & 1\\
	0 & 0 & 0\\
\end{bmatrix}$\\

$\begin{bmatrix}
	1\;x_1 & -1\;x_2 & 0\;x_3\\
	0\;x_1 & 0\;x_2 & 1\;x_3\\
	0\;x_1 & 0\;x_1 & 0\;x_3\\
\end{bmatrix} = \begin{bmatrix}
	0\\
	0\\
	0\\
\end{bmatrix} \rightarrow x_1-x_2=0 \; ; \; x_3=0 \rightarrow$ Mögliche Lösung: b1 = $\begin{bmatrix}
	1\\
	1\\
	0\\
\end{bmatrix}$(da neu, n=max)\\

Step 4: Basisvektor $b_2$ und $b_3 \rightarrow (A- \lambda_2 I)^n = 0$ für $n=\{1,2\}$\\
\textcircled{2} rref(A+3*identity(3))\\
\hspace*{2cm}$\begin{bmatrix}
	1 & 0 & 0\\
	0 & 0 & 1\\
	0 & 0 & 0\\
\end{bmatrix}$\\

$\begin{bmatrix}
	1\;x_1 & 0\;x_2 & 0\;x_3\\
	0\;x_1 & 0\;x_2 & 1\;x_3\\
	0\;x_1 & 0\;x_2 & 0\;x_3\\
\end{bmatrix} = \begin{bmatrix}
	0\\
	0\\
	0\\
\end{bmatrix} \rightarrow x_1=0 \; ; \; x_3=0 \rightarrow$ Mögliche Lösung: $\begin{bmatrix}
	0\\
	10\\
	0\\
\end{bmatrix}$ (Nicht $b_2$, n$\neq$max!!)\\

\textcircled{3} rref((A+3*identity(3))\textasciicircum2)\\
\hspace*{2cm}$\begin{bmatrix}
	1 & 0 & 1\\
	0 & 0 & 0\\
	0 & 0 & 0\\
\end{bmatrix}$\\

$\begin{bmatrix}
	1\;x_1 & 0\;x_2 & 1\;x_3\\
	0\;x_1 & 0\;x_2 & 0\;x_3\\
	0\;x_1 & 0\;x_2 & 0\;x_3\\
\end{bmatrix} = \begin{bmatrix}
	0\\
	0\\
	0\\
\end{bmatrix} \rightarrow x_1 + x_3=0 \rightarrow$ Mögliche Lösungen: $\begin{bmatrix}
	0\\
	1\\
	0\\
\end{bmatrix} \; ; \; b3 = \begin{bmatrix}
	-1\\
	0\\
	1\\
\end{bmatrix}$(da neu, n=max)\\

Step 5: Fehlender Basisvektor $b_2=N_2*b_3$\\
(A+3*identity(3))*[-1;0;1]\\
\hspace*{2cm}$\begin{bmatrix}
	0\\
	1\\
	0
\end{bmatrix}$\\

Step 6: Zusammensetzen
$T=\begin{bmatrix}
	b_1 & b_2 & b_3
\end{bmatrix}$ sowie $J=T^{-1}*A*T$\\{}
[1,0,-1;1,1,0;0,0,1]$\rightarrow$ t\\
\hspace*{2cm}$\begin{bmatrix}
	1 & 0 & -1\\
	1 & 1 & 0\\
	0 & 0 & 1\\
\end{bmatrix}$\\
t\textasciicircum-1*a*t\\
\hspace*{2cm}$\begin{bmatrix}
	2 & 0 & 0\\
	0 & -3 & 1\\
	0 & 0 & -3\\
\end{bmatrix} \rightarrow $ Jordan-Form!